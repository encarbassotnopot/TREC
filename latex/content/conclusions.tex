\chapter{Conclusions}
Amb aquest treball tenia com a objectiu entendre i comparar diversos algoritmes d'ordenació.

Considero que no només he assolit els objectius que em vaig plantejar, sinó que a mesura que desenvolupava el treball anava ampliant els meus coneixements sobre programació. Aquest camp sempre m'havia interessat i aquesta no ha estat la primera vegada que m'hi endinso, però ha estat la primera vegada que ho he fet amb un objectiu clar i amb una excusa per passar-me unes quantes hores donant voltes a un algoritme que aleshores encara no entenia.

Durant la recerca i el redactat del treball m'he adonat de com n'és d'important l'efectivitat d'un algoritme d'ordenació.
Si amb els coneixements que tenia quan vaig començar aquesta tasca hagués dissenyat un algoritme d'ordenació hauria fet una cosa similar al \textit{bubble sort}, potser a l'\textit{insertion}. Aquests dos algoritmes són els dos algoritmes més lents que he treballat.

Fent aquest treball, a més d'aprendre sobre programació i tenir el meu primer contacte amb l'algorítmica he aprofitat l'avinentesa per tocar altres disciplines, també digitals, de les que vull fer gala.
Per exemple, els gràfics de les Torres de Hanoi els he fet a mà en SVG.
Les taules, els gràfics de rendiment i el càlcul de l'exponent $p$ no els he fet en cap full de càlcul, ho he fet en Notebooks de Jupyter amb Pandas i Matplotlib.
Finalment aquest document no ha passat per un sol processador de text sinó que està fet en \LaTeX. Malgrat que d'entrada això pugui semblar un hàndicap, he de confessar que una vegada l'he entès no m'ha portat cap problema, ans el contrari, m'ha facilitat extraordinàriament la tasca.

Ara sí, per concloure d'una vegada per totes aquest treball, només em queda afegir que els recursos que es disposen per fer un TREC són limitats, especialment el temps.
Amb això no vull suplicar la misericòrdia del tribunal, sinó obrir la porta a una possible segona part d'aquest treball.
No l'escriuré jo, però considero que aquest camp és prou ampli per poder continuar la recerca on l'he deixada jo o bé començar de nou analitzant altres algoritmes o estudiant-ne altres aspectes.