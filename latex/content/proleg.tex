\chaptertoc{Pròleg}

\noindent{\large Estimada filla meva...}

Sóc filòleg de formació, amb set anys d'experiència com a docent de secundària, i pagès de vocació, amb 18 anys treballant com a viticultor en una finca familiar agrària. Els mateixos anys que té la meva primera filla, l'Eina. Als ulls de la qual sempre seré un boomer.

I és que jo, a la seva edat (una expressió que m'avergonyeix escriure perquè pensava que jo mai la utilitzaria per comparar-me amb els meus fills, com van fer els meus pares amb mi) feia els treballs amb una Hispano Olivetti. Quan la tinta s'acabava, aixecaves la tapa i amb el dit feies girar la bobina fins a fer passar tota la cinta d'una banda a l'altra del carret. Si volies afegir alguna imatge als teus treballs, compraves revistes, les retallaves amb tisores i les enganxaves amb Imedio, aprofitant aquell moment que el destapaves per fer unes quantes inspiracions fortes que afegissin una mica d'emoció a les nostres vides, abans de fer forats amb la perforadora i posar un modern fastener o unes clàssiques anelles.

Segurament m'hauria pogut imaginar molt abans que estàvem en dues dimensions irreconciliables, quan amb pocs anys, ja em solucionava els problemes que tenia amb el mòbil. Quan veia que m'acostava massa la pantalla als ulls, em preguntava -Vols que et posi la lletra més gran?. Jo no sabia ni tan sols que es pogués fer. O quan em queixava que no m'hi cabien més fotos i que havia de començar a esborrar-ne em deia, per a gran esgarrifança i temor meu, -Per què les guardes al telèfon? Posa-les al núvol!. I així va seguir. Ja no sabem canviar-nos el mòbil si no el configura ella. No podem veure la televisió si es desconfigura el mòdem de la fibra. Allò de picar la tele o dir que és culpa d'ells està molt mal vist per les noves generacions. Fins i tot la Thermomix avui s'ha de configurar amb la clau wifi, perquè t'actualitzi les receptes.

Sóc usuari actiu de les xarxes. Tinc Facebook (com totes les persones de provecta edat), Twitter, fins i tot, Instagram. A TikTok no hi he entrat, per ara. Diàriament utilitzo el correu electrònic, el Whatsapp i, per si de cas és insegur o ens espia, com asseguren de tant en tant els mems que rebo, ja m'he descarregat Telegram i Signal, preventivament, per no quedar-me incomunicat. Pràcticament no poso els peus a la meva entitat bancària perquè faig tota l'operativa on-line i ja he entès, que encara que no tingui els diners en l'apunt bancari d'una llibreta, els meus diners són igual de volàtils. I em declaro un analfabet digital.

Estic envoltat i utilitzo una tecnologia que no entenc. I quan he fet algun esforç per entendre-la m'he sentit desplaçat. Crec que la tecnologia i jo ja estem amortitzats mútuament. Ella em facilita uns serveis i jo els consumeixo. És un pacte en el qual tots dos tenim allò que volem. Ella un consumidor submís i jo unes comoditats que em fan creure que puc ser original escrivint un missatge de 140 caràcters i un fotògraf artístic retocant una imatge. Però que no falli res, perquè si surt algun missatge que no m'espero, he de fer venir l' Eina per preguntar-li què carai significa.

I aleshores ho soluciona. I ja no prova d'explicar-m'ho perquè la generació Z i els boomers som com l'oli i l'aigua. Però, com a pare, no puc renunciar al mínim control parental que se'ns suposa, que segur que s'ha recollit en els drets de la infància o en qualsevol declaració similar de les Nacions Unides. I miro què fa. Sense entendre res. Perquè allà on jo hi veia un full en blanc, com el que posava a la Hispano Olivetti, i que omplo amb lletres amb la màgica possibilitat d'esborrar-les, corregir-les, canviar-les... ella fa aparèixer un llenguatge ocult, ple de caràcters impossibles i, per descomptat, illegibles. És com un subtext. Com si caigués el decorat d'un teatre i quedés la tramoia al descobert, amb operaris fumant i efectes especials sense glamur. Com si a un cotxe nou i llampant, li traiessis la carrosseria i quedés la mecànica al descobert. Amb la diferència, que aquí la tramoia que aguanta la cosa, la mecànica que fa córrer la màquina, és un llenguatge.

I aquest llenguatge subjacent a les aplicacions, també té un llenguatge que fa les aplicacions interactuïn amb el seu suport i amb altres aplicacions. I per això calen els algoritmes: un conjunt d'instruccions per resoldre un problema o executar una ordre. En aquests algoritmes és on la tecnologia es juga bona part de la seva raó de ser. I jo, sense entendre-ho, sento una admiració profunda per la meva filla que és capaç de fer un treball explicant-ho. Un treball sobre fórmules matemàtiques, àlgebres i números, que són presents en la nostra quotidianitat i que preferim ignorar però, com fa el Neo a Matrix, l'Eina s'ha atrevit a prendre's la pastilla vermella per posar-lo al descobert. 

És així, com es passa pàgina i s'avança. Quan la generació que et succeeix és millor que la teva. Enhorabona Eina per la capacitat d'anar més enllà i qüestionar-t'ho tot.

\begin{flushright}
	Gil Coma Masana \\ Pare de l'Eina
\end{flushright}