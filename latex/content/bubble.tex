\chapter{Bubble sort}

\section{Com funciona?}
El \textit{bubble sort} compara cada element de la llista amb el següent (d'esquerra a dreta), canviant-ne la posició si el segon és més gran que el primer.
Aquest procés es repeteix tantes vegades com faci falta per ordenar la llista.
Sempre movem endavant l'element més gran de manera que, progressivament, al final del vector queden ordenats els elements més grans.
El nom \textit{bubble sort} li és donat precisament per com els elements més grans floten cap a la superfície, com si es tractés de bombolles.
És un algoritme extremadament lent.

\section{Implementació}
\lstinputlisting[language=Python]{../scripts/bubble.py}
\vspace{1em}

\begin{minipage}{\textwidth}
	\section{Rendiment}
	\noindent
	\makebox[\textwidth][c]{
		\begin{tabular}{lrrrrrrr}
\toprule
{} &   125  &   250  &   500  &    1000 &     2000 &     4000 &  8000 \\
\midrule
\textbf{Temps (s)} & 0.0283 & 0.2121 & 1.6764 & 14.0336 & 116.9414 & 966.9290 &   (-) \\
\bottomrule
\end{tabular}

		}
	\vspace*{1em}
	\begin{center}
		\resizebox{\textwidth}{!}{\includesvg{bubble}}
	\end{center}
\end{minipage}